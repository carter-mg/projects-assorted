\documentclass{article}
\usepackage{fitch}
\usepackage{amsmath}
\usepackage{amssymb}
\usepackage{graphicx}
\usepackage{fitch}

\begin{document}
    \title{"Proof of Churches"}
    \author{Carter Garrett}
    \date{26 January 2024}
    

    \maketitle
    \begin{center}
        A personal interest project, explorations of logical relations in contemporary theology using First Order Logic. 
    \end{center}

    \newpage
    \section{Exploration 1}
    
    Let C represent the attribute of Church. 
    Let the D relationship represent $'derives from'$.
    \newline
    \begin{fitch}
        \fa \exists x\exists y \exists z((C(x) \wedge C(y)) \wedge \lnot(x = y) \wedge (C(z) \rightarrow (z=x \vee z = y))) :$PR$\\
        \fa \exists x(D(x, l)) :$PR$ \\
        \fa \exists y(D(y,d)) :$PR$\\
        \fa C(x) \rightarrow (D(x,l) \vee D(x,d)) :$PR$\\
        \fa (C(x) \wedge  D(x,l)) \rightarrow (P(x)) :$PR$\\
        \fj \lnot(D(x,l)) \rightarrow D(x,d) :$PR$\\
        \fa \fh (C(a) \wedge C(b)) \wedge \lnot(a = b) \wedge (C(c) \rightarrow (c=a \vee c = b)) :$AS$ \\
        \fa \fa \fh D(a, l) :$AS$ \\
        \fa \fa \fa \fh D(b, d) :$AS$ \\
        \fa \fa \fa \fa (C(a) \wedge C(b)) :\wedge$E$ 7 \\
        \fa \fa \fa \fa (C(a)) :\wedge$E$ 10\\
        \fa \fa \fa \fa (D(a,l)) \vee D(a,d) :\rightarrow$E$ 4, 11\\
        \fa \fa \fa \fa C(a) \wedge D(a,l) :\wedge$I$ 8 , 11\\
        \fa \fa \fa \fa P(a) :\rightarrow$E,$ 11, 13 \\
        \fa \fa \fa D(b,d) \rightarrow P(a) :\rightarrow$I$9 - 14\\
        \fa \fa D(y,d) \rightarrow P(x) :\exists15 \\
        \fa (C(a) \wedge C(b)) \wedge \lnot(a = b) \wedge (C(c) \rightarrow (c=a \vee c = b)) \rightarrow D(y,d) \rightarrow P(x) :\rightarrow$I$7 - 16\\
        \fa \therefore (C(x) \wedge C(y)) \wedge \lnot(x = y) \wedge (C(z) \rightarrow (z=x \vee z = y)) \rightarrow D(y,d) \rightarrow P(x) :\rightarrow$I$7 - 16\\
    \end{fitch}

    If the two religious entities are at odds, it seems to imply an absolute kind of opposition. No possibility for  mutual success. Granted, 'success' and 'opposition'are extremely vague terms.

    \newpage

    \begin{center}
        Demonstration of logical principle of explosion.
    \end{center}

    \begin{fitch}
        \fa A :$PR$ \\
        \fj \lnot A :$PR$ \\
        \fa \bot :1,2 \\
        \fa \therefore B :$X$3\\
    \end{fitch}

    \newpage 
 
    (On 2 Nephi 2:11)
    
    \begin{fitch}
        \fh \lnot O \rightarrow \lnot (G \wedge E):$PR$\\
        \fa \fh O :$AS$\\
        \fa \fa G \wedge E :\rightarrow $E$ 1,2\\
        \fa \therefore G \wedge E \\

       
    \end{fitch}
    (Skipped double negation step)
    \newline
    A very simple implication from the verse. 
\end{document}